
%% NOTE: this does not describe the latest version of tralics.
\documentclass[11pt]{report}
\usepackage[T1]{fontenc}
\usepackage[latin1]{inputenc}
\usepackage{fancyvrb}
\usepackage{amssymb,amsmath}
%\usepackage[frenchb]{babel}
\usepackage{verbatim}
\let\thebar|
\DefineShortVerb{\|}
%\usepackage{a4}
%\usepackage{descaux}
\usepackage{RR}

\def\tralics{\textsc{tralics}}

\DefineVerbatimEnvironment{verbatim}{Verbatim}
{fontsize=\small,listparameters={\topsep0pt }}

\textwidth16cm
\textheight22cm
\evensidemargin0cm
\oddsidemargin0cm
\parskip 2pt plus 2pt

\SaveVerb{BA}|\|\def\BA{\UseVerb{BA}}
\SaveVerb{OV}|{|\def\OV{\UseVerb{OV}}
\SaveVerb{FE}|}|\def\FE{\UseVerb{FE}}
\SaveVerb{rend}|rend|
\SaveVerb{bhi}|<hi>|
\SaveVerb{yesstar}|star='yes'|
\SaveVerb{verb}|\verb|
\DefineShortVerb{\+}
\SaveVerb{VE}+|+\def\VE{\UseVerb{VE}}
\SaveVerb{DU}|$_|\def\DU{\UseVerb{DU}} %$
\UndefineShortVerb{\+}

\providecommand\xmllatex[1]{}

\def\Han{H�n Th\xmllatex{&\#x1ebf;}{\'{\^e}} Th�nh}

\def\mybox#1{\par\medskip\noindent\fbox{\large \texttt {<#1>}}}


\RRtitle{Tralics : un traducteur \LaTeX\ vers XML}
\RRetitle{Tralics: a \LaTeX\ to XML translator}
\RRprojet{MIAOU}
\RRtheme{4}
\RRauthor{Jos\'e Grimm}
\URSophia

\bibliographystyle{plain}


\begin{document}

\RRresume{En l'an 2001, il a �t� d�cid� que le format pivot du rapport
  d'activit� serait XML, c'est pour cela qu'on a �crit un traducteur de
  \LaTeX\ vers XML, en Perl, puis en C++.

  On d�crit dans ce document comment utiliser le traducteur.}

\RRabstract{In 2001, it was decided that the scientific part of Inria's annual
  report uses XML as intermediate language. We have hence written a \LaTeX\ to
  XML translator, first in Perl, then in C++.
  We explain here how to use this software}


\motcle{Latex, XML, HTML, MathML,PostScript, Pdf, syntaxe, s�mantique}
\keyword{Latex, XML, HTML, MathML, PostScript, Pdf, syntax, semantics}
\makeRT
%\makeRT

\chapter{Introduction}
This document describes yet another \TeX-to-something translator; in this
first chapter we give some motivations, and in the next ones, we explain the
software. Since 1982, when Knuth released \TeX, the situation has evolved a
lot: processors became faster, memory cheaper, new languages like Pdf have
been invented, the Web plays a great role in information exchange, etc.  On
the other hand, \TeX\ still uses the technology of the seventies, and their
paradigms. For instance, the only way to introduce color in a dvi document is
via |\special|s, so that, on one hand, the document becomes device dependent
(when dvi stands for device independent), and on the other hand, special
precautions are required in the source document if order for the colors to
stack properly. It is difficult to put dvi documents on the Web, because they
are not standalone (on the other side of web, there must be a, at least
partial, \TeX\ distribution), and appear as raw data (a dvi file cannot be
indexed like an HTML page or a Pdf document, because there are no key words in
it).

% \def\pluriel#1#2{#1s #2s}\def\singulier#1#2{#1 #2}
% \def\sp{\iftrue\expandafter\pluriel\else\expandafter\singulier\fi}
% TESTSP\sp{toto}{titi}

There have been some tentatives to extend \TeX, some of them available only
in commercial softwares, others (like $\epsilon$-\TeX\ or Ml\TeX) are part of
the standard distributions, but nobody makes use of them. There are two major
softwares, pdf\TeX\ and $\Omega$ that have enough features to make them
attractive (see also \cite{Hagen2001}, which deals mainly with NTS). For
instance, $\Omega$ removes some internal limits (number of registers, etc.),
allows full Unicode input, bi-directional typesetting, etc. As a result, dvi
files, TFM files, etc., are incompatible, and you need to upgrade all the
tools (xdvi, dvips, dvidvi, etc). On the other hand, pdf\TeX\ uses standard
\TeX\ input and produces Pdf files, which have become a de facto standard on
the Web. However, there are some informations in a Pdf file that do not exist
in a dvi file: the paper size, title, author, subject, keywords, etc. A
special syntax is required for them (on the other hand, hyperlinks, bookmarks,
etc., are often obtained by automatically redefining some \LaTeX\ commands).

There is no equivalent for producing HTML; perhaps because \TeX\ The
Program is too difficult to modify, but mostly because \TeX\ is designed to
produce (directly or not) black and white pixels on a sheet of paper (or a
screen), and not structured documents. There are a lot of translators from
\LaTeX\ to HTML, some of them use \TeX\ behind the scene, others are written
in C (very efficient), or Perl (easy to customize), etc. All these translators
use the structuring mechanisms of \LaTeX.


  A software like |latex2html| has nearly one hundred different options
(some of them are not described in the on-line help), and can generate
code for some different versions of HTML: currently 2.0, 3.0, 3.2, and 4.0.
In general, people want to split the document into several HTML pages (but
should we split on |\sections|, |\subsections|, or elsewhere?), and link these
pages (should navigation buttons be on top of the page, or also on the
bottom, should the ``next'' button be an image with an arrow, or some
text, in English, or in the local language ?), putting a table of contents
somewhere (on the main page only?, at the top or bottom of other pages?).

This document describes \tralics, a \LaTeX\ to XML translator. The main
assumption we make is that the XML document produced by \tralics\ will, sooner
or later, be transformed into something readable by a human being
(dvi, Pdf, HTML, etc), under a form not specified a priori. For instance, we
may assume that the document will be converted to both HTML and Pdf, and we
may have to answer a lot of questions like:  what is the paper size for the
Pdf case? at which level to split the HTML? should we put the table of
contents at the beginning or the end the Pdf document? what about the HTML
case?, etc. 

Notice that the real difference between HTML and Pdf is not the way data are
organized in the file, it is how they can be exploited by a human being: 
in the case of a PostScript document, printed on some sheets of paper, what we
expect from an index and a table of contents is a page number (in some cases,
line numbers can be useful too), and this of course implies that the pages
have numbers; in the case of an HTML document, that is generally read on a
screen, we expect the document to react positively in case the size of the
window changes (thus, there is no a priori formatting of the text), and the
table of contents (the index, etc.) should contain active links. Since Pdf can
be read in both ways, we need both: like PostScript, typesetting is done once
and for all, pages are numbered, and the table of contents contains page
numbers; like HTML the table of contents contains active links. 

Being able to produce a document that can be converted to both HTML and Pdf is
not a problem for \tralics: in fact, no typesetting is done, no table of
contents nor index is generated, but there is enough information in the XML
file for an XML processor to do so. Thus, when \tralics\ sees |\chapter|, it
has to output |<chapter>| instead of expanding the command and executing more
basic actions (that could be: finish the current page, flush all floating
material still in memory, switch to an odd page, change locally the page
style, reset a lot of counters, add some white space at the begin of the page,
typeset ``Chapter two'' and the name of the chapter in a huge bold font, add a
new entry to the table of contents, etc). 


\section{The name of the game}
We shall assume that the reader is familiar with HTML and Pdf (if he is not,
he will not understand some arguments presented in this introduction), and
with XML (The Extensible Markup Language (XML) is the universal format for
structured documents and data on the Web, see |http://www.w3.org/XML/|). In
fact, XML is a markup language, in which you can say: |<chapter>| starts here
and ends there, but no significance is given to the word ``chapter'', it could
have been |<chapitre>|, |<Kapitel>| (or the same in Japanese). There are
nevertheless some keywords: for instance |<fo:block>| is a reserved word in
the FO (formatting objects) namespace, which plays the same role as boxes in
\TeX\ (\tralics\ will never generate such objects), |<xsl:when>| is a reserved
word in the XSLT namespace (XSLT is a natural way to convert an XML object to
another one, \tralics\ will never generate such objects), and |<m:msubsup>| is
a reserved word in the MathML namespace (it defines constructs  like
$x_2^3$); note that \tralics\ \emph{will} generate such keywords.

We assume also that the user is familiar with \TeX-the-program and
\TeX-the-language (if no confusion can arise, we just say \TeX) and the
extensions $\Omega$, pdf\TeX, and \LaTeX. Note that  $\Omega$ and pdf\TeX\ are
two programs, that are based on the same sources as \TeX, that can process a
\TeX\ document and produce a dvi file, although pdf\TeX\ is mainly used to
generate Pdf files, and $\Omega$ can be used to generate right-to-left text 
inside left-to-right text (thus you can insert some arabic text into some
english text). In \tralics, we have implemented only basic \TeX, no
extensions, except this one: you can say |\char N|, and |\chardef\x N| where N
is a 16bit integer (instead of a 8bit integer). This is to allow the use of
any Unicode character in the output: a character like � can be represented 
in the output as |�| (iso-8859-1 encoding), or |&#233;| (Unicode numeric
value), or |&eacute;| (Unicode name); on the other hand, the |\oe| ligature
can be given only as |&oelig;| or |&#339;|. You can say |\char339| instead of
|\oe| in your input (this is very useful in case the input is encoded in
UTF-8, see later for details).


Just as pdf\TeX\ is an extension of \TeX-the-program, \LaTeX\ is an extension
of \TeX-the-language, that defines a great many number of commands, most of
them being explained in the documentation. In the next paragraph, we shall
discuss the |\textit| command: this command is defined as: call the font
changing command |\itshape| on the argument (this implies handling italic
correction, and there is somewhere a comparison of the first parameter of the
current font against zero). On the other hand, |\chapter| is just a small \TeX\
command (it's only 20 tokens long, but calls commands like |\@chapter| which
are much longer); it's not even defined in the \LaTeX\ kernel, but the
semantics is clear. Note that |\chapter| calls |\newpage|, which starts a new
page, essentially by executing |\penalty-10000| after a |\par\vfil|.

\tralics, our \LaTeX\ to XML translator, is able to expand every \TeX\ macro,
but knows about \LaTeX\ semantics, thus stops expansion before its comes to a
dead-end (the first parameter of a font is the slant, so that |\textit| will
check whether the old and new font or both slanted or both unslanted;
inserting a |\penalty| of value $-10000$ causes \TeX\ to break a line or a
page there. Any other penalty is only a hint, thus completely ignored by
\tralics.


\section{The philosophy of the translation}
Let's assume that we have to translate |\textit{foo}|. What \TeX\ does is to
introduce a font switch command in the dvi file. If the current font scheme
is: T1 encoding, computer modern type face, eleven point size, and if |\texit|
means: switch to italics, normal weight, normal family, then the font name
might be |ecti1095|. In order to process this file, an external process
(xdvi, dvips, etc) must know, hence read, at least the font metrics (possibly
other files, like metafont sources, or bitmap files). On the other hand, a Pdf
file contains everything that is needed to process the file. This limits the
possibilities of representation of the font: either something understood by
Adobe Acrobat or bitmaps. Using bitmaps is not a good idea, because it fixes
the resolution (and Adobe Acrobat is still unable to render then properly on a
screen). Since the |ecti1095| does not exist in Type 1 format, some people use
|cmti10| instead (it's a similar font, at ten point, but it can be
rescaled). There are two problems for the non-english speaking people: since
this is a 7bit font, words like � caract�re accentu� � do not consist only of
letters. One consequence is that it forbids hyphenation, another one is that
the word cannot be found using the search facility of Acrobat (there is also
the problem that the characters `�' and `�', french opening and closing quotes,
do not exist in this font). One important point is that our translator
translates the argument of the font command, in a way independent of the font
(it may depend on some global settings, for instance).

The HTML solution is a first step in the good direction: first a standard way
of encoding character is used (namely iso-10646, which is the same as
Unicode), and for many characters, the Unicode name can be used (for instance 
|&gamma;| for the Greek letter $\gamma$). In XML, any Unicode character can be
used, by its number or its name. The translation of the example would be
|<i>foo</i>|. However, the translation of |{\bf a \textit{b} c}|,
cannot be |<b>a<i>b</i>c</b>|, because this gives a bold italic b. Thus, a
correct translation of |\textit| is not trivial. In \tralics, we could decide
a priori that |<b>| is bold, |<i>| is italic, etc., independently of the
context. But then, we need a name for bold-italic, so that using the \TeX\
name |ecti1095| is a solution not worse than other. However, a better scheme
would be to generate something like
\begin{verbatim}
<font family='fam-x' series='series-y' shape='shape-z' size='size-t'>...</font>
\end{verbatim}
which maps \LaTeX\ font scheme (the font encoding is not given here, it is
the same for the whole document). The expression above is just a name, like 
|ecti1095|, but we could decide that |<font shape='it'>| is a shorthand for
a |<font>| element where all the missing attributes are inherited,
and  |<font shape='it' defaut='yes'>| is the same where all missing fields
have the default value.
 
\section{Parameterisation and DTD}
The DTD (Document Type Definition) is the specification of the syntax of the
result of the translation. Consider an example where the DTD sayus that the
toplevel element can contain |<div0>|, that can contain |<div1>| that can
contain |<div2>|, etc.  If the document to convert is a book, then |\part|
matches |<div0>|, |\chapter| matches |<div1>| and |\subsection| matches
|<div3>|. On the other hand, an article has no chapter, nor part, so that
|\subsection| matches |<div1>|. Assume that the DTD allows a |<p>| in a |<b>|
in a |<p>| and a |<i>| in a |<b>|. How is \tralics\ assumed to translate
|{\bf a \textit{b} c}|?  If the current element is |<b>|, it may output a |<i>|
element, but DTD does not say if this gives normal italic or bold italic.
In fact, the DTD does not say that its gives italic at all.

In some cases, the translator cannot emit a |<i>| element (perhaps because we
are in a |<p>|, and |<i>| can be used only in a |<b>| inside a |<p>|), or
cannot emit a |<i>| with the given argument (for instance, the argument is a
character string, and |<i>| needs a |<j>|). For all these reasons, translation
is independent of the DTD, and an error might be signalled later (by
xmllint, for instance). Remember: translation should respect the
\emph{semantics}, while the DTD offers only syntax.



This does not mean that the translator produces syntactically correct
documents. In some cases, the  document has constraints that
cannot be given in the DTD, for instance: the document must contain A, B, C
and D, in this order, but one of them is optional. There are some constaints
that can be expressed in SGML but not in XML (for instance, you want to allow
a footnote in a paragraph, a paragraph in a footnote, so that this could give
a P in a F in a P: ok, a F in a P in a F: refused). What we propose is a
translator that is independent of the DTD. You can, if you want, call
automatically xmllint (or any other XML checker) from \tralics. 

Hence, there is a way to tell \tralics\ to produce special effects. The easy
case is to make |\foo{bar}| translate into |<foo>bar</foo>|.
It is also possible to ask for |\font{A}{B}| to be translated into 
|<font name='A'>B</font>|.  Then you can say |\def\textit{\font{it}}|,
and you have a working definition for |\textit|.


It is possible to parameterise \tralics\ in order to get one or the other
behavior, but the situation is much different for |\itshape|. In fact, a
command such as |\itshape| says: from now on, the current font should use
an italic shape. Thus \TeX\ and pdf\TeX\ will generate, wherever needed,
a reference to the desired font (for instance, the dvi might contain something
like: this page is in italics). It is possible to put 
|{\itshape| in the middle of a paragraph and |}| in the middle of the next
one, and this gives two paragraphs with a font change in it, so that the XML
we generate will be of the form |<p>a<it>b></it></p>| followed by
|<p><it>c></it>d</p>|. 

\chapter{All commands}
\def\showcmd#1{\fbox{{\large\tt\textbackslash #1}}}
\def\showenv#1#2{\fbox{{\large\tt\textbackslash begin\{#1\}#2...\textbackslash end\{#1\}}}}
\section{Accents}
\showcmd{a} In \LaTeX, the |\a| command is a special hack, in that it 
allows to construct an accent
that is otherwise disabled by some environments like |\tabbing|. As long 
as you do not try to play
with macro-expansion, you should see no difference with \tralics. 

\showcmd{accent}
In \TeX, you can say |\accent XAY|, where X is character code, A a list of 
assignments, Y a character. The result 
is to put character at position X in the current font onto character Y (of the current font, but A can 
contain a font change). This mechanism is not available in tralics: the translator will generate a reference to
some Unicode character. In tralics, |\accent| will signal an error, and ignore the arguments.

The following accents are recognised by tralics. The characters generated are defined in iso-latin1 and iso-latin2.
\begin{itemize}
\item circonflex accent. Available on a, e, i, o, u, c, g, h, j, s, w, y, upper or lower.
\item acute accent. Available on a, e, i, o, u, y, c, l, n, r, s, z (upper and lower) and g lower.
\item grave accent. Available on a, e, i, o, u (upper and lower).
\item accent trema. Available on a, e, i, o, u (upper and lower).
\item macron accent. Available on a, e, i, o, u (upper and lower).
\item dot accent. Available on  c, e, g, z (upper and lower).
\item cedilla accent. Available on c, k, l, n, r, s, t (upper and lower) and g  (upper).
\item breve accent. Available on g, a, u (upper and lower).
\item check accent. Available on c, d, e, l, n, r, s, t, z (upper and lower).
\item tilde accent. Available on a, o, i, u, n (upper, lower)
\item umlaut accent. Available on o (lower).
\item ogoneck accent. Available on a, e, i, u, upper and lower.
\item b t r accents. Not available.
\end{itemize}
Example
\begin{verbatim}
 \^a \'a \`a \"a \=a \.e \c c \u a \v c \~ a \H o \k a  \b t \d t \r t 
\end{verbatim}
 \^a \'a \`a \"a \=a \.e \c c \u a \v c \~ a \H o \k a  \b t \d t \r t 

\showcmd{AA} \showcmd{aa}. These are defined as \verb|\r A| and |\r a|
in \LaTeX, but as a character reference in tralics.

\showcmd{l} This generates the l-stroke character.

\showcmd{\$} this command generates a dollar sign.

\showcmd{,} This command generates a space, in fact |&nbsp;|.

\showcmd{\_} This command generates an underscore character.

\showcmd{\&} This command generates an ampersand character.

\showcmd{\#} This command generates a sharp sign character.

\showcmd{\%} This command generates a percent sign character.

\showcmd{S} This command generates a section sign characer.
This is the same as |�|.

\showcmd{ } The backslash-space and backslash-newline command both generate
a space character.

\showcmd{o} \showcmd{O} These command generate a slashed o. This is
exactly the same as  |�| or |�|.

\showcmd{-} This command produces nothing (in \LaTeX, |\-| generates a discretionary hyphen).

\showcmd{@} This command produces nothing. In \LaTeX, it generates sometimes
a weird error (if the |\catcode| of the at-sign is wrong, and you say something
like |\@foo|) about an invalid |\spacefactor|. See the description of  
the |\spacefactor| command, in order to see why assigning this variable to
1000 would not change the XML output in a significant way.


\showcmd{space} This is a command whose expansion is a space character.
OK ???

\showcmd{lq} This is the same as a left quote (a command whose expansion
is a left quote).

\showcmd{rq} This is the same as a right quote  (a command whose expansion is a right quote).

\showcmd{lbrack} This is the same as a left bracket  (a command whose expansion is a left bracket).

\showcmd{rbrack} This is the same as a right bracket  (a command whose expansion is a right bracket).

\showcmd{endgraf} This is the same as |\par| (a command that has the same meaning as the |\par| primitive).

\showcmd{bgroup} This is the same as an open brace (a command that has the same meaning as an open brace)

\showcmd{egroup} This is the same as close brace (a command that has the same meaning as an close brace)

\showcmd{empty} This is a user defined command whose body is empty.

\showcmd{null} The result of this command is an empty hbox. Tralics constructs
an XML element whose printed value is void. 

\showcmd{active} This is number 13. Technically, it is |\chardef|, but who cares. This is in general only used in cases like |\catcode 10=\active|.

\showcmd{inputlineno} Returns the current input line number.

\showcmd{wlog} Undefined. a definir. The argument is printed in the log file.

Difference between \underline{punderline underbar} and \underbar{punderline underbar}.

newbox a definir...

\showcmd{xmllatex}
This command could be defined as 
\begin{verbatim}
  \newcommand\xmllatex[2]{#2}
\end{verbatim}
However, \tralics\ ignores the second argument and converts the first in a special 
verbatim mode: each token is printed in the result as follows: if the token is a
command, its name is printed (without the initial backaslash), otherwise it is 
a character that is  pinted as is. This may generate invalid XML.
Example:
\begin{verbatim}
\def\entity#1{\xmllatex{\&#1;}{}}
\def\strok#1{\entity{#1strok}}
\def\Han{H�n Th\xmllatex{&\#x1ebf;}{\'{\^e}} Th�nh}
\end{verbatim}
\def\entity#1{\xmllatex{\&#1;}{}}
\def\strok#1{\entity{#1strok}}
\def\Han{H�n Th\xmllatex{&\#x1ebf;}{\'{\^e}} Th�nh}
The expansion of |\Han| is \Han. This is the only way provided in \tralics\ for double
accents. The result of |\strok{l}| is empty in \LaTeX\, the XML quantity |&lstrok;| 
otherwise. By the way, this is the same as |\l|.


This is not yet implemented...
\begin{verbatim}
\def\null{\hbox{}} % hacker
\end{verbatim}

\showenv{xmlelement}{\{name\}} This allows you to construct a named xml element

\showcmd{AddAttToLast} \showcmd{AddAttToCurrent} These two commands take
two arguments, say A and B, and add an attribute-value pair to the
last or current XML element. No check is made on the syntax of the attribute name. No check is made on the existence of an attribute of the same name.
Example
\begin{verbatim}
\begin{xmlelement}{foo}
\begin{xmlelement}{subelt}
text
\end{xmlelement}
\AddAttToLast{sub-att}{att-val}
\AddAttToCurrent{foo-att}{att'value}
\end{xmlelement}
\end{verbatim}
should produce something like
\begin{verbatim}
<foo foo-att='att&apos;value'>
<subelt sub-att='att-val'>
text
</subelt></foo>
\end{verbatim}

backash
\section{Conditionals}
There are different tests in \TeX. For instance
\begin{verbatim}
\ifnum\count0=0 A\else B\fi
\end{verbatim}
compares the value of the count register number zero to the integer zero; if true
that A is evaluated, otherwise B is evaluated. More precisely, \TeX\ remembers that 
it is in an |\if| construct. If the test is true, expansion and evaluation proceeds
as usually, until an |\else| or a |\fi| is seen. In the first case, all tokens between
the |\else| and the |\fi| are read at high speed. If the test is false, \TeX\ reads 
all tokens at high speed until finding a |\fi| or an |\else|. In the second case,
evaluation proceeds normally. Reading tokens at high speed means that any token whose
meaning is |\fi| finishes a conditional, any token whose meaning is |\if|, |\ifnum|, 
etc., open a new test. Said otherwise, if |\A|, |\B| and |\C| are some user defined 
commands that do not use conditional commands, then the following
\begin{verbatim}
\let\IFNUM\ifnum\let\FI\fi\let\ELSE\else
\ifnum\count0=0 \A \else \ifnum\count0=1 \B \else \C \fi\fi
\IFNUM\count0=0 \A \ELSE \IFNUM\count0=1 \B \ELSE \C \FI\FI
\end{verbatim}
produce the same code. 

You can say something like
\begin{verbatim}
\def\Atrue{\let\ifA\iftrue}\def\Afalse{\let\ifA\ifalse}
\ifnum\coun0=0 \Atrue \else \Afalse\fi
\end{verbatim}
but not
\begin{verbatim}
\ifnum\coun0=0 \let\ifA\iftrue \else \let\ifA\iffalse\fi
\end{verbatim}
The reason is the following: if the test is true, what we get after the |\else|
is |\let|, |\ifA| (whose meaning is |\iftrue|) then |\iffalse| and |\fi|. Two more
|\fi|s are needed. In the case where the test is false, the situation is different.
In the case where |\ifA| is undefined, we have one missing |\fi|, if |\ifA| 
is |\iftrue|, |\iffalse| or something like that, there are 3 missing |\if|s, and 
if |\ifA| is |\fi| it will terminate the |\ifnum| conditional. After that, 
the |\iftrue| is evaluated and |\ifA| will terminate it, and then the |\iffalse| 
is evaluated, and all three conditional are balanced.

The following code is wrong
\begin{verbatim}
\def\FI{\fi}
\ifnum\count0=0 A\else B\FI
\end{verbatim}
This is because the |\FI| is a |\fi| in case the test if false, but not if the
test is true. On the other hand, this code
\begin{verbatim}
\def\A#1\fi{okok #1\fi}
\ifnum\count0=0 \A something \else B\fi
\end{verbatim}
is perfectly valid code: if the test is false, the |\fi| finishes the 
conditional, otherwise, it is read by |\A|, and |\A| produces its own |\fi| to 
terminate the conditional. 


Consider the following example
\begin{verbatim}
\count0=0
\ifnum\count0=0 A\else B\fi
\ifnum\count0=0 A\fi
\count0=1
\ifnum\count0=0 A\else B\fi
\ifnum\count0=0 A\fi
\end{verbatim}

This is what \TeX\ puts in the log file, when |\tracingall| has been evaluated.
\begin{verbatim}
{\count}
{\ifnum}
{true}
{the letter A}
{\else}
{\ifnum}
{true}
{\fi}
{\count}
{\ifnum}
{false}
{the letter B}
{\fi}
{\ifnum}
{false}
\end{verbatim}
For the first test, you see the |\else|, for the second test you see the |\fi|.
For the third test, you see the |\fi|, and for the last, you see nothing

The log file of \tralics\ is a bit different. It will contains the following:
\begin{verbatim}
[466] \count0=0
{\count}
+scanint for \count->0
+scanint for \count->0
\end{verbatim}
Here, you see that \tralics\ has seen the |\count| command, and that this command has
read an integer, namely zero, then a second integer (it's also zero).
\begin{verbatim}
[467] \ifnum\count0=0 A\else B\fi
+\ifnum
+scanint for \count->0
+scanint for \ifnum->0
+scanint for \ifnum->0
+iftest true
+\else
+\fi
\end{verbatim}
Here, you can see that \tralics\ has seen a |\ifnum| which read two integers. The
first integer is the value of |\count0|, this means that |\count| must fetch
its argument first. In the example, all three numbers are zero, the test is true.
We do not show the trace for the other cases, just notice that the trace contains
always a |+\fi|, and a |+\else| if the input contains a |\else| token.

\showcmd{newcount} \showcmd{newdimen} \showcmd{newlength}
Each of these commands takes an argument and defines it globally as a 
new counter, dimenion or length. More precisely, |\newcounter\foo| is like
|\countdef\foo N|, where N is the current allocation number for counters; it
is incremented at each call. All allocation numbers start at 10.

\showcmd{newcounter} This command takes as argument a character string,
and defines a \LaTeX\ counter. For instance, |\newcounter{foo}| defines
the counter |\c@foo|. Moreover |\thefoo| is globally defined to be 
|number\c@foo|. The command takes a final optional argument. If it is for 
instance |[bar]|, then |bar| must be a counter, and incrementing |bar| via
|\stepcounter| will reset |foo| to zero.

ET OPT ARG? FAIRE QUELQUE CHOSE.

\showcmd{value} This command takes a counter name as argument, and returns
the value of the counter. More precisely, |\value{foo}| is the same as
|\c@foo|.

\showcmd{setcounter}, \showcmd{addtocounter}, \showcmd{stepcounter}.
These commands take a counter as argument, say |foo|, and modify globally
the counter. The first two commands take a number T as second argument, and 
set the value of the counter to T, or increments it by T. On the other hand,
|\stepcounter| increments the value by 1, and may reset some other counters.
Morew precvisely
\begin{verbatim}
  \setcounter{foo}{T}
  \addtocounter{foo}{T}
  \stepcounter{foo}
\end{verbatim}
is the same as
\begin{verbatim}
  \global\advancc@ T\relax
  \global\c@foo T\relax
  \global\advance\c@foo 1\relax
\end{verbatim}
abd if the calc package is loaded, this is changed to
\begin{verbatim}
  \@calc{\global\advancc@}{T}
  \@calc{\global\c@foo}{T}
\end{verbatim}



\section{Expandable Commands}

\showcmd{[}, \showcmd{]} Both these commands expand to |$$|. In \LaTeX, 
the closing bracket inserts a |\ignorespaces| command, but in \tralics, 
displaymath is always outside a paragraph. The opening bracket inserts 
|\nointerlineskip\makebox[.6\linewidth]|, that is ignored by tralics.


\showcmd{(}, \showcmd{)}. Both these command expand to just a single
dollar.
    
\showcmd{csname} This command reads all tokens, up to |\endcsname|,
and constructs the control sequence whose name is formed by the tokens
just read. If this produces an undefined command, the command will be (locally)
defined as |\relax|. For instance, |\csname foo\endcsname| is the same as
|\foo| (but does nothing is |\foo| is undefined).

This allows to construct control sequences that are empty or
contain funny characters; for instance |\csname$ \endcsname| %$
constructs the control sequence whose name is a dollar sign followed
by a space. 

Commands are fully expanded, but non-expandable commands are forbidden. Example
\begin{verbatim}
\def\fooaux{bar}\def\foo{\fooaux}\let\gee\endcsname
\csname\foo\gee
\csname x\par\endcsname
\end{verbatim}
Here, the first |\csname| generates the |\bar| token, the second |\cname|
complains that |\par| is forbidden (the expansion is empty, and an extra 
|\endcsname| error will be signalled). In the following example, we construct 
a command whose name ends with a backslash. This is achieved by calling |\string|
on |\\|, this gives a double backslash;  the first backslash is removed by calling 
|\gobble| on this result, using |\expandafter| to alter evaluation order.
\begin{verbatim}
\def\gobble#1{}
\csname foo\expandafter\gobble\string\\\endcsname
\end{verbatim}
Note that the command constructed by |\csname| is never undefined


\chapter{The configuration file}
After all arguments are read, tralics constructs the |Tdir| variable: this is
the place where some file can be found. This variable is obtaine
\begin{itemize}
\item as the value of the |-dir| argument of the command,
\item as the value of the envirnonment variable |TRALICSDIR|,
\item as the value of the environment variable |RAWEBDIR|,
\item or as some default value (which could be related to the author's
home dir). 
\end{itemize}

It is possible to parameterise \tralics\ via a configuration file.
The file is searched first
in the current working directory, then in the subdirectory `src' 
of the |Tdir| variable, finally in the home directory of the user, or in a 
subdirectory `src' of it. In case where the environment variable |HOME|
is set, it is assumed to be the home directory. Otherwise, if the variable
|USERNAME| is set and \tralics\ is compiled on windows, it is assumed that
the home directory is |//d/username|. The name of the configuration file
is |.tralics_rc| (without the initial dot on windows).




The same configuration file can be used for more than one document type. Thus
the configuration file contains two kinds of information: what is the document
type? what should be done for a this document type?

The syntax of the configuration file is the following: 
\begin{itemize}
\item There are lines that define a keyword. The keyword is the first element
  on the line, it is followed by an equals sign and a value (spaces are
  allowed before and after the equals sign). 
\item There are lines that define a block. A block starts with something like
|BeginBlock|, and ends with |EndBlock|. In fact, the end of a block is defined
by a line that starts with the three letters ``End''. 
\item All other non trivial lines must start with white space (a space
  character or tabulation).
\item Empty lines and lines starting with a comment may appear everywhere
\item A comment is introduced by a |%| or a |#| sign. In some cases, it is
possible to put a comment at the end of a line. This will be explained later.
\end{itemize}

Let's start with a simple example.
\begin{Verbatim}
Type=\documentclass
BeginType article
DocType= article article.dtd
End

BeginType book
DocType= book article.dtd
BeginMacros
  \def\foo{\bar}
EndMacros
End
\end{Verbatim}
On the first line, we have a keyword, named |Type| with value |\documentclass|.
Then we have two blocks, named |Type|, these blocks have a value, |article|
and |book|. In each block is a keyword, |DocType|, and the value is character
string formed by two words. In the second block, there is a subblock |Macros|,
that contains some text (here a macro definition). 

Note that you can put a comment (or anything else) after the keyword that
follows |BeginType|. This keyword should contain only letters. In the same
fashion, you can put a comment after the keyword that follows |Type=|.
Comments are allowed after the second keyword following |DocType|. You can put
a |%| on the line that contains |\def\foo|, but this is not a comment:
it will be considered by the translator as a comment.



\paragraph{The document type}
Finding the type of a document requires three actions. First, we compute the
potential type. This is the value of the |-type| argument of the translator if
given, the value of the |Type| keyword otherwise, and if this |Type| is not
given, or is |\documentclass|, it will be the argument of the |\documentclass|
declaration of the main document. After that, the list of all defined types is
computed. In the example given above, there are two types: 
|article| and |book|).

If the potential type is a known type, it becomes the real type. Otherwise,
we look at aliases. These are defined like this :
\begin{Verbatim}
BeginAlias
  article report 
  slides inriaslides foiltex
End
\end{Verbatim}
The idea is: if the first type on the line is defined, and the potential
type is one of the other terms, then the first term will become the real type.
Thus |report| will be the same as |article|, while |inriaslides| and |foiltex|
will remain potential types (since |slides| is undefined).
Note: a type contains only letters, while an alias may contain also digits.

In case of failure, another check is made, by removing final letter from
the potential type. Otherwise, the first type that is defined is chosen.

Consider the following exemple: the document class holds is |ra2001|,
there are two defined types, |RR| and |RA|, the aliases are those indicated
aboved, plus these:
\begin{Verbatim}
BeginAlias
  RR toto# ra2001
  RA ra toto ra2001%etc
        
End
\end{Verbatim}

The translator might print something like this:
\begin{Verbatim}
Read configuration file src/.tralics_rc
Seen \documentclass ra2001
Config type \documentclass
Potential type is ra2001
Defined type: RA
Defined type: RR
Alias article undefined
Alias slides undefined
Alias RR does not match ra2001
Alias RA does not match ra2001
Trying type ra
Alias article undefined
Alias slides undefined
Alias RR does not match ra
Potential type ra aliased to RA
\end{Verbatim}
This means that the document type is |RA|.

After this operation is done, every line of the configuration file
that are is not in the current |Type| block is discarded (Noting is done here
if there is no |Type| block, hence no type).

\paragraph{Main document configuration}
From now on, we assume that we have found a document type, and that is ts
|article|. This means that the configuration file contains something like:
\begin{Verbatim}
BeginType article
DocType= article article.dtd
DocAttrib = year \specialyear
DocAttrib = test "foo"
morestuff...
End
\end{Verbatim}
The |DocType| keyword explains the doctype to use. In fact, the translator
is now ready to generate the following XML code:

\begin{Verbatim}
<?xml version='1.0' encoding='iso-8859-1'?>
<!DOCTYPE article SYSTEM 'article.dtd'>
<!-- translated from latex by tralics 0.0-->
<article ???>
\end{Verbatim}

We have put some question marks in |<article ???>|. This means that
the |<article>| element has some attributes, that can be set anywhere in
the \LaTeX\ document. But you can say |DocAttrib = x y| in the configuration
file. This will add |x='y'| to the attributes of the document.
If the string |y| is encloded in double quotes, the value is used unchanged.
Otherwise, it starts with a backslash. It this case, it is a \LaTeX\ command
that will be evaluated later. The effect is the same as saying
\begin{Verbatim}
  \AtBeginDocument{\addattributestodocument{x}{\y}}
\end{Verbatim}
Exception: if |y| is |\specialyear|, the value is either the value of the
paramter |-year|, or, in the case of the RA, the year extracted from the
|\documentclass|. With the |DocAttrib|s given above, the next line of the
resulting document might be :

\begin{Verbatim}
<article year=2001 test="foo">
\end{Verbatim}

\subsection{A non trivial example}
Let's consider the following non trivial example. It computes the 30 first
prime numbers. Knuth pretends
that the trial division was executed 132 times. We modified the code so that 
it computes and show this number.
\paragraph{loops}

\begin{verbatim}
\countdef\td 4 \td=0
\newif\ifprime \newif\ifunknown
\newcount\n \newcount\p \newcount\d \newcount\a
\def\primes#1{2,~3% assume that #1 is at least 3
  \n=#1 \advance\n by-2 % n more to go
  \p=5 % odd primes starting with p
  \loop\ifnum\n>0 \printifprime\advance\p by2 \repeat}
\def\printp{, % we will invoke \printp if p is prime
  \ifnum\n=1 and~\fi % this precedes the last value
  \number\p \advance\n by -1 }
\def\printifprime{\testprimality \ifprime\printp\fi}
\def\testprimality{{\d=3 \global\primetrue
  \loop\trialdivision \ifunknown\advance\d by2 \repeat}}
\def\trialdivision{\a=\p \global\advance\td by 1\divide\a by\d
  \ifnum\a>\d \unknowntrue\else\unknownfalse\fi
  \multiply\a by\d
  \ifnum\a=\p \global\primefalse\unknownfalse\fi}

The first thirty prime numbers are \primes{30}. trial division macro was
expanded \the\td\ times
\end{verbatim}

If you say |\tracingall|, you will see in the log file the whole job done by
\tralics. For instance, associated to |\countdef\td 4 \td=0|, you will see:
\begin{verbatim}
{\countdef}
+scanint for \countdef->4
{\td}
+scanint for \td->0
\end{verbatim}
The rules for scanning an integer in \TeX\ may be hard to undersand. Thus,
whenever \tralics\ has scanned an integer it shows in the log file the value,
and the destination (here |\countdef| and |\td|). We use use |\td| as a
counter to count the number of calls to |\trialdivision|. If you allocate a
counter via |\newcount| its value is zero. Here we must zero it by hand.

This is the log for the expression |\newcount\n|.
\begin{verbatim}
{\countdef \n=1548}
\end{verbatim}
Note that, in \LaTeX, you would see something like |\n=\count28|. Here the
value 1548 is an index in the |eqtb| table (this is irrelevant).

When the first prime number has been computed, (namely 5), the macro |\printp|
is called. Here is the log of it:
\begin{verbatim}
\printp->, \ifnum \n =1 and~\fi \number \p \advance \n by -1 
+\ifnum
+scanint for \ifnum->28
+scanint for \ifnum->1
+iftest false
+\fi
+scanint for \number->5
{\advance}
+scanint for \advance->-1
\end{verbatim}
The effect of this procedure is first to add a comma and a space to the
current XML tree (these two characters are in a temporary buffer). Then comes
a test. The macro |\ifnum| looks for an expression like $A<B$, $A>B$ or
$A=B$. It needs two numbers; as you can see, these numbers are 28 and 1, and
the test is false. It adds nothing to the current XML tree. After that
|\number| evaluates |\n|, and adds its value to the current tree (in the
temporary buffer). This buffer is flushed by any command that might non-text
to the current tree, here |\advance| (this is due to the fact that |\advance|
is a member of the family of commands that can be preceeded by |\global|).
The net effect of |\printp| is hence to add comma, space, et 5 to the XML tree,
decrement the counter |\n|.

In the case where the whole code is in a group, the old value of |\n| is saved
on the save stack (but only the first time). Then, at the end of the group,
you can see something like:
\begin{verbatim}
+stack: restoring integer value 1548 0
\end{verbatim}
Note that the definition of |\n| is global (as well as |\a|, so that you lose
the construct |\a'e|).


In plain \TeX, we have the following definition:
\begin{verbatim}
\def\loop#1\repeat{\def\body{#1}\iterate}
\def\iterate{\body \let\next\iterate \else\let\next\relax\fi \next}
\let\repeat=\fi % this makes \loop...\if...\repeat skippable
\end{verbatim}

This is changed, in \LaTeX\ to
\begin{verbatim}
\def\loop#1\repeat{\def\iterate{#1\relax\expandafter\iterate\fi}%
  \iterate \let\iterate\relax}
\let\repeat\fi
\end{verbatim}

In \tralics, we use a variant of the \LaTeX\ code, where |\iterate| is a
private macro. Note that, if you define a loop inside a loop, the inner
|\iterate| will kill the outer |\iterate|. For this reason, the inner loop
must be in a group, and some variables must be changed globally (|\printrue|
et |\td|). The following is printed in the log file when the closing brace is
seen in |\testprimality|. 
\begin{verbatim}
{end group character}
+stack: level - 5
+stack: restoring \ifunknown
+stack: restoring integer value 1551 0
+stack: restoring \iterate
+stack: restoring integer value 1550 0
+stack: level boundary
\end{verbatim}
The first line is printed by the translator: it sees a command that asks for
popping the stack. It then decrements the current level (which is 5), then unwinds
the stack. As you can see, two integer values were modified in the group, namely
 |\a| and |\d|, and a command |\ifunknown|. The process of unwinding the stack stops
when a level boundary has been seen.
In the case where $n=9$, which is not a
prime number, the value of |\ifprime| is changed to false by the line 
|\ifnum\a=\p \global\primefalse|. The trace for these lines is 
\begin{verbatim}
+scanint for \ifnum->9
+scanint for \ifnum->9
+iftest true
\primefalse->\let \ifprime \iffalse 
{\global\let}
{\let \ifprime \iffalse}
\end{verbatim}

The \LaTeX\ trace for |\global\primefalse| is the following:
\begin{verbatim}
{\global}
\primefalse ->\let \ifprime \iffalse 
\end{verbatim}
As you can see, when \TeX\ sees |\global\let|, it prints only the |\global|
into the log file, while \tralics\ prints more. In some cases, the trace of
\TeX\ is hard to understand. Consider for instance :
\begin{verbatim}
|\def\a{\b}\expandafter\let\a\c|
\end{verbatim}
You get:
\begin{verbatim}
{\expandafter}
\a ->\b 
{\let}
\end{verbatim}
You have no idea here of what is expanded after what, and what is defined. In
\tralics, you will see:
\begin{verbatim}
{\expandafter \let \a}
\a->\b 
{\let}
{\let \b \c}
\end{verbatim}
This shows that |\let| is expanded after |\a|, and that |\let| defines |\b| to
|\c|.

\paragraph{ifthenelse}
The constructions of the |ifthen| package are implemented, but differently
from the current \LaTeX\ version. The main algorithm is the following :
If you say |\ifthenelse{test}{A}{B}|, then one of |A| or |B| is executed, this
depends on the value of |test|, which is analysed in the following way. Let T
be the first token of |test| that is neither a space, nor |\relax|. If this is
a normal token, the |\ifnum| is executed (more precisely, two numbers and an
operator are
read, and compared), this gives a result, true or false, and can leave some
tokens in the main token list, see later. If |T| is |\isodd|, then  |\ifodd|
is executed on the argument (i.e., an argument is read, pushed back on the main
list with a |\relax| as separator, a number is read, and checked for parity).
There are little chances that |\ifodd{\pageref{someref}}| gives a result.
If |T| is |\lengthtest|, then |\ifdim| is called on the argument (i.e., an
argument is read, pushed backed on the main list with |\relax| as terminator,
two dimensions and an operator are read, and the dimensions are compared).
If |T| is |\boolean|, and argument is read. It is an error if the expansion
contains non-characters. Assume that the result is `foo'. The test is true if
the token |\iffoo| is equivalent to |\iftrue|, false otherwise. Finally, if
|T| is |\equal|, two arguments are read, fully expanded, and compared for
equality. In any case, after we have found the value of the test there can
remain some unread tokens on the main list.

In the case where the first token |T| is |\(|, the procedure is called
recursively, all remaining tokens are read, until finding a closing |\)|.
In case where the first token is |T|, the result is negated. 

In the case where the first unread token, that is neither space nor relax
is |\or| or |\and|, the following is done : In the case |\or|, and the
condition is true, nothing happens, otherwise, the remaining is evaluated for
truth value. In the case |\and|, if the
condition is false, nothing happens, otherwise, the remaining is evaluated for
truth value. Otherwise, remaining tokens are ignored. In the following
example, each line should expand into |ok|. 

\begin{verbatim}
\ifthenelse{\not\boolean{true}\and\boolean{true}}{bug}{ok}
\ifthenelse{\not\boolean{true}\and\boolean{false}}{bug}{ok}
\ifthenelse{\not\boolean{false}\and\boolean{true}}{ok}{bug}
\ifthenelse{\not\boolean{false}\and\boolean{false}}{bug}{ok}

\ifthenelse{\not\(\boolean{true}\and\boolean{true}\)}{bug}{ok}
\ifthenelse{\not\(\boolean{true}\and\boolean{false}\)}{ok}{bug}
\ifthenelse{\not\(\boolean{false}\and\boolean{true}\)}{ok}{bug}
\ifthenelse{\not\(\boolean{false}\and\boolean{false} good ?\)}{ok}{bug}

\ifthenelse{\boolean{false} \or \boolean{false} or \boolean{true}}{ok}{bug}
\end{verbatim}


If |\tracingcommands| is positive, you can see how the machanism works.
The following is taken from the LaTeX companion.
\begin{verbatim}
\newcommand{\QU}[2]{%
\ifthenelse {
  \(\equal{#1}{ENG} \and \equal{#2}{yes} \)
     \or
  \(\equal{#1}{FRE} \and \equal{#2}{oui} \)
}{``OK''}{``not OK''}}
You agree \QU{ENG}{yes} or don't \QU{ENG}{no}. 
D'accord \QU{FRE}{oui} ou pas \QU{FRE}{non}. 
\end{verbatim}

The trace for the first |\QU| is:
\begin{verbatim}
{ifthenelse \(}
{ifthenelse \equal}
{ifthenelse equal true}
{ifthenelse \and continuing}
{ifthenelse \equal}
{ifthenelse equal true}
{ifthenelse-> true}
{ifthenelse \or skipping}
{ifthenelse-> true}
\end{verbatim}
In this case, we have first a |\(|, followed by an |\equal|. We print the
value of the |\equal|, since |\equal| forces macro-expansion, so that, if
|tracingmacros| is true, there may be a lot of lines between these two lines.
After that comes and |\and|, and we continue (because the value was true).
Then, we see an |\equal|, and return true. This is the truth value of the 
subexpression |\(...\)|. It is followed by an |\or| that does nothing.

This is the trace of the second |\ifthenelse|. It is of the form (true and
false) or (false and something).
\begin{verbatim}
{ifthenelse \(}
{ifthenelse \equal}
{ifthenelse equal true}
{ifthenelse \and continuing}
{ifthenelse \equal}
{ifthenelse equal false}
{ifthenelse-> false}
{ifthenelse \or continuing}
{ifthenelse \(}
{ifthenelse \equal}
{ifthenelse equal false}
{ifthenelse \and skipping}
{ifthenelse-> false}
\end{verbatim}

\subsubsection{The calc package}
It's of course available only if you load the package. In this case,
|\setcounter|, |\addtocounter|, |\setlength| and |\addtolength| are redefined
so that you use infix arithmethics. Example:
\begin{verbatim}
\newcounter{Ac}
\setcounter{Ac}{(1+2)*(3+4)-20}\theAc=1
\addtocounter{Ac}{(1*2)+(3*-4)+(34/7)}\theAc=-9
\end{verbatim}
In the example above, all computations were done using integer values.

Evaluation is from left to right, unless you use parentheses. Example.
\begin{verbatim}
\newcounter{hours}\newcounter{minutes}
\newcommand{\printtime}{%
   \setcounter{hours}{\time/60}%
   \setcounter{minutes}{\time-(\value{hours}*60)}
   \thehours h \theminutes min}The time is \printtime.
\setcounter{hours}{3* \real{1.6} *\real{1.7}}
\end{verbatim}
Here 3 times is 4.8, rounded to 4, and 4 times 1.7 is 6.8, rounded to 6,
and the result is 6. On the other hand, multiplying 1.6 by 1.7 gives 2.72,
after multiplication by 3, this gives 8.16, rounded to 8. There is no way to
achieve the multiplication in this order, using the |calc| package.
On the other hand, if you says |A+B*C|, the calc package may evaluate the
multiplication first. This is not explained in the documentation, neither in
the \LaTeX\ companion, which gives the example above (without the
parentheses). In our implementation, the parentheses are required in this
example. 

In the case of dimensions, you have to take some care.
\begin{verbatim}
\newlength{\Bc}
\setlength{\Bc}{(1cm+2cm)*(3+4)-200mm} \the\Bc=28.4526pt
\setlength\Bc{\the\Bc*\ratio{25.4pt}{722.7pt}} \the\Bc=0.99985pt
\Bc=1in \setlength\Bc{\the\Bc * 100 / 7227} \the\Bc=0.99998pt
\Bc=1in \setlength\Bc{\the\Bc * \real{ 0.001383700013837}} \the\Bc=1.00018pt
\Bc=1cm \setlength\Bc{\the\Bc / \real{28.452755}} \the\Bc=0.99985pt
\Bc=1cm \setlength\Bc{\the\Bc * \ratio{254pt}{7227pt}} \the\Bc=0.99985pt
\Bc=1in \setlength\Bc{\the\Bc / \ratio{7227pt}{100pt}} \the\Bc=1.00018pt
\end{verbatim}
The last two expressions give some arithmetic error, and a completely wrong
result in \LaTeX. In \tralics, we solved the problem, by dividing numerator
and denominator by an adequate power of two. In any case, if the denominator 
is greater than $2^{31}/10 sp$, or 3276pt, there can be an overflow in LaTeX. 
(the denominator is $B$ if you multiply by $A/B$, and $A$ if you divide).

\subsection{titlepage}
The title page is customised by saying something like this
\begin{Verbatim}
BeginTitlePage
  \makeRR <RRstart> "" " type = 'RR'"
  alias \makeRT "123" " type = 'RT'"

  <UR> -
  \URSophia ?+<UR>
  \URRocquencourt ?+<UR>
  alias \URRocq
  \Paris ?<UR> <Rocquencourt>
  \URRhoneAlpes ?+<UR>
  \URRennes ?+<UR>
  \URLorraine ?+<UR>
  \URFuturs ?+<UR>

  \RRtitle <title> "pas de titre"
  \RRetitle <etitle>  "no title"
  \RRprojet <projet> "pas de projet"
  \RRtheme <theme>  "pas de theme"
  \motcle <motcle> "pas de motcle"
  \keyword <keyword>  "no keywords"
  \RRresume p<resume> "pas de resume"
  \RRabstract p<abstract> "no abstract"
  \RRauthor + <author> <auth> "Pas d'auteurs"
  \RRdate <date> A"\monthyearvalfr"
  \RRNo <numero> "????"

End
\end{Verbatim}

Here every line follows a pattern
\begin{Verbatim}[numbers=left]
mac elt str str
alias mac str str
mac elt str
mac +elt elt str
elt -
mac ?+elt
mac ?elt elt
mac mac str
mac elt mac
alias mac
\end{Verbatim}

If you say |\makeRR <RRstart> "" " type = 'RR'"|, this will define 
a command |\makeRR| that takes no argument, and constructs the titlepage.
If you say |alias \makeRT "123" " type = 'RT'"|, then |\makeRT| is an
alternate way to create this title page.


If |tracing_io| is true, then every source line is printed in
the log file. It is preceeded by the line number in square brackets.

\section{etc}
optional star ignored in |\newcommand|.

\paragraph{Trace}
There is an integer variable |\tracingrestores| that controls

When \tralics\ sees a closing brace, and tracing-restores is true, it
prints in the log file the status of the variables to be restored.
For some reasons, 
\begin{verbatim}
{\let\foo\relax\tracingall\let\bar\relax}
\end{verbatim}


\section{empy}

There is a questin about whether or not the argument of |\begin|
or |\end| should be expanded. Condider the following code.
\begin{verbatim}
\newenvironment{Foo}{startfoo}{endfoo}
\def\endBar{endbar}
\def\foo{Foo}
\begin{\foo}\def\foo{Bar}
\end{\foo}
\def\Bar{Foo}
\begin{\foo}
\end{\Bar}
\end{verbatim}



For efficiency reasons, when \tralics\ reads a dimension, and a unit, if the
first token is a character then a second token is read. Said otherwise,
after |\def\foo\foo| and |\dimen0=1w\foo\relax|, \tralics\ expands |\foo| and
signals an error about the syntax of |\foo|, followed by a missing unit error,
while \TeX signals errors in the other order. 





\bibliography{lx}



\end{document}
