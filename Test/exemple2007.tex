%%% Exemple de rapport d'activit�s 2007
%%% Thiswas exemple2007.tex, comments removed

\documentclass{ra2007}
\newcommand{\calX}{\mathcal{X}}
\newcommand{\toto}{X}


\project{EXEMPLE}{ExemplE}{Algebraic Systems for Research and Industry} 
\theme{com}
\isproject{oui}
\UR{RhoneAlpes}
 
\begin{filecontents+}{exemple_all2007.bib}
@inproceedings{refer1,
author =	{First Author},
title =		{Compilation of LOTOS Abstract Data Types},
booktitle =	{Proceedings of foo},
year =		{1989},
editor =	{Son T. Vuong},
pages =		{147--162},
publisher =	"nh",
month =		dec,
}

@inproceedings{refer2,
author =	{Second Author},
title =		{Some title},
booktitle =	{Proceedings of the First International Conference},
year =		{1998},
editor =	{Bernhard Steffen},
publisher =     "sv",
address =       {Berlin},
series =        "lncs",
volume =        {1384},
pages =         {68--84},
month =         mar,
note =          {Full version available as INRIA Research Report~RR-3352},
url =		{http://www.inria.fr/rrrt/rr-3352.html},
}

@inproceedings{refer3,

author =	{Third Author},
title =		{SVL: a Scripting Language for Compositional Verification},
booktitle =	{Proceedings of the 21st IFIP},
year =		{2001},
editor =	{Myungchul Kim and Danhyung Lee},
pages =	        {377--392},
organization =  {IFIP},
publisher =	{Kluwer Academic Publishers},
month =		aug,
}

@INPROCEEDINGS{MB04,
        author = "Bronstein, Manuel",
        title="Efficient Algorithms for Linear Ordinary Differential Equations",
        editor = "Alonso, Marc and Sendra, Robert",
        booktitle = "Cuarto Encuentro de Algebra Computacional y Aplicaciones",
        year = 2004,
        month = sep,
        pages = {159-163},
        organisation = "Universit\'e de Alcala de Henares"
}

@InCollection{cite1,
  author =       "Arsac, Olivier and Dalmas, St\'ephane and Ga{\"e}tano, Marc",
  booktitle =    "Computer-Human Interaction in Symbolic Computation",
  series = "Texts and Monographs in Symbolic Computation", 
  title =        "Algorithm Animation with {AGAT}",
  chapter = "8",
  pages =   "163-177",
  publisher = "Springer-Verlag",
  year =      "2007"
}
@Misc{cite2,
  title = {Mathematical Markup Language {(MathML)} 1.0 Specification},
  url = {http://www.w3.org/TR/REC-MathML/},
  month =     apr,
  year =   2007,
  key = {mathml},
  note =      {{W3C} Recommendation}
}






\end{filecontents+}
\begin{document}
\maketitle


\nocite{cite1}

\begin{moreinfo}
  Texmex is a common project with CNRS, University of Rennes~1 and INSA. The
  team has been created on January the 1\textsuperscript{st}, 2002 and became an
  INRIA project on November the 1\textsuperscript{st}, 2002.
\end{moreinfo}

\begin{module}{composition}{en-tete}{}
   \begin{catperso}{Head of project-team}
       \pers{Christine}{Eisenbeis}{Chercheur}{INRIA}[Research Associate (CR) Inria][Habilite]
   \end{catperso}

   \begin{catperso}{Administrative assistant}
       \pers{Nathalie}{Gaudechoux}{Assistant}{INRIA}[Secretary (SAR) Inria]
   \end{catperso}
   
   \begin{catperso}{Research scientist Inria}
     \pers{Fran�ois}{Thomasset}{Chercheur}{INRIA}[Research Director (DR) Inria]
   \end{catperso}
   
   \begin{catperso}{Reserch scientists (external)}
       \pers{Jean}{Louchet}{Chercheur}{AutreEtablissementPublic}[Ing. en chef Armement (CR)]
       \pers{Jean-Marie}{Rocchisani}{Chercheur}{UnivFr}[Universit� Paris XIII]
   \end{catperso}
   
   \begin{catperso}{Visiting scientist}
       \pers{Moussa}{Lo}{Visiteur}{UnivEtrangere}[AUF Grant/ Gaston Berger University, Saint-Louis, Senegal, from March 1st till August 31]
   \end{catperso}
   
\end{module}

\begin{module}{presentation}{overall-objectives}{Overall Objectives}

  The explosion of the quantity of numerical documents raises a problem.
  \begin{itemize} 
  \item Firt item
  \item Second item
  \end{itemize}
\end{module}

\begin{module}{presentation}{Highlights}{Highlights}
This year, a new module appears.....
\end{module}

\begin{module}{fondements}{description}{Document Description and Metadata}
  \begin{motscle}
    Low-level Descriptor, Metadata
  \end{motscle}

\begin{participants}
\pers{Ioannis}{Emiris},
\pers{Jean}[de]{La Fontaine}[1621-1695],
\pers{Cecil Blount}{De Mille}
\end{participants}

\begin{glossaire}
\glo{Content-based indexing} {the process of extracting from a
  document (here a picture) compact and structured significant visual 
  features that will be used and compared during the interactive
  search.}
\end{glossaire}

\begin{moreinfo}
Common activity with LoveGeom project. 
\end{moreinfo}

\paragraph{xx}
Usually subspace identification is a one step procedure.

\subsubsection{First subsection}
Due to the increasing broadcasting of digital video content
finding copies in a large video  database has become a critical new issue.

\paragraph{toto}
The article describing and comparing output-only and input/output covariance-driven subspace 
identification methods (see 2005 activity report) has been published \cite
{cite2}.

\paragraph{titi}
The article describing the general framework.
an IEEE journal \cite{cite2}.

\paragraph{tutu}
The \emph{cosmad} toolbox, 
see module~\protect\moduleref{EXEMPLE}{logiciels}{modal}.

\subsubsection{second subsection}
Text of second subsection
\paragraph{tata}
Text...
\end{module}

\begin{module}{}{ident}{Identification}
...
\end{module}


%%%%%%%%%%%%%%%%%%%%%%%%%%%%%%%%%%%%%%%%%%%%%%%
%%% Modules dans la section domaines d'applications


\begin{module}{domaine}{panorama}{Panorama}

\begin{motscle}
telecommunications,
multimedia,
biology,
health,
process engineering,
transportation systems,
environment
\end{motscle}
\end{module}

\begin{module}{}{telecom}{Telecommunication Systems}
Modules should not be empty.
\end{module}

\begin{module}{}{logembarque}{Software Embedded Systems}
Modules should not be empty.
\end{module}



%%%%%%%%%%%%%%%%%%%%%%%%%%%%%%%%%%%%%%%%%%%%%%%%%%%%%%%%%%%%%%%%%%%%%%%%%%%%%%
%%% Section logiciels.
%%%%%%%%%%%%%%%%%%%%%%%%%%%%%%%%%%%%%%%%%%%%%%%%%%%%%%%%%%%%%%%%%%%%%%%%%%%%%%

\begin{module}{logiciels}{modal}{Hyperion Software} 
\begin{participants}
  \pers{Jos�}{Grimm}[correspondant],
  \pers{Laurent}{Baratchart}[projet Miaou],
  \pers{Fabien}{Seyfert} [projet Miaou]
\end{participants}
\begin{motscle}
Conformance Testing,
TGV,
Lotos
\end{motscle}

See also the web page
\htmladdnormallink{\url{http://www-rocq.inria.fr/scilab/}}
{http://www-rocq.inria.fr/scilab/}.

Alternate versions (note order of arguments)
\href{http://www-rocq.inria.fr/scilab/}{\url{http://www-rocq.inria.fr/scilab/}}.

You can simplify this to 
\url{http://www-rocq.inria.fr/scilab/}.
\end{module}



%%%%%%%%%%%%%%%%%%%%%%%%%%%%%

\begin{module}{resultats}{tralics}{Tralics: a LaTeX to XML Translator}

(...) On a le d�veloppement suivant:
\[ \forall f\in C^\infty\left(\left[-\frac{T}{2};\frac{T}{2}\right]\right),
   \forall t\in \left[-\frac{T}{2};\frac{T}{2}\right],
   f(\tau) = \sum_{k = -\infty}^{+\infty} e^{2i\pi\frac{k}{T}t} \times
   \underbrace{\frac{1}{T}
               \int_{-\frac{T}{2}}^{\frac{T}{2}} f(t) e^{-2i\pi\frac{k}{T}t} dt
              }_{a_k = \tilde{f}\left(\nu = \frac{k}{T}\right)}
\]
et puisque (...)

\end{module}

%%%%%%%%%%%%%%%%
%%% Section contrats 

\begin{module}{contrats}{edf}{EDF}
\begin{participants}
\pers{Christ�le}{Faure},
\pers{Jean-Charles}{Gilbert}, 
\pers{Jos�}{Grimm}
\end{participants}
ici des math en html 
... $y=x^2$ ...

et ici on g�n�re une image pour le web  
\begin{displaymath}
\sum_{0}^{\infty} y = x^4
\end{displaymath}
...

\end{module}





\begin{module}{international}{national}{National Actions}

\subsubsection{Incitative Action  FIABLE}
%%% liste de participants associee
\begin{participants}
\pers{Christ�le}{Faure},
\pers{Jean-Charles}{Gilbert}
\end{participants}
blabla

\subsubsection{Incitative Action  MOUAI}
\begin{participants}
\pers{Jean-Charles}{Gilbert}, 
\pers{Jos�}{Grimm}
\end{participants}
blabla<w references  \refercite{refer1,refer2,refer3}
\end{module}


\begin{module}{}{europe}{Actions Funded by the EC}
\subsubsection{Projet LTR TURLU EP-9134867}
A citation \footcite{MB04} using the \verb=\footcite= command
\subsubsection{R�seau TMR RATA}
blabla
\end{module}

\begin{module}{diffusion}{animation}
  {Animation de la Communaut� scientifique}
A citation \refercite{refer1} using the \verb+\refercite+ 
(major publications of the Team).
\end{module}



\begin{module}{}{enseignement}{Teaching}
...
Note. Optional arguments like [htbp] to the figure environment will be ignored.

\begin{figure}
\begin{center}
\includegraphics{IMG/uneimage}
\end{center}
\caption{An example of a map reconstructed by using geometrical methods in detecting landmarks}
\label{fig:completemap}
\end{figure}
...
citation \cite{cite2} using \verb+\cite+ for publications of current year.

...
\end{module}

\loadbiblio

%%%%% fin du document
\end{document}
